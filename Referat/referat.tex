\documentclass[12pt,a4paper]{article}
\usepackage[utf8]{inputenc}
\usepackage[romanian]{babel}
\usepackage{amsmath}
\usepackage{amsfonts}
\usepackage{amssymb}
\usepackage{graphicx}
\usepackage{geometry}
\usepackage{hyperref}
\usepackage{enumitem}
\usepackage{fancyhdr}

\geometry{margin=2.5cm}
\pagestyle{fancy}
\fancyhf{}
\rhead{Cuadrice Dublu Riglate în Arhitectură}
\lfoot{\thepage}

\title{\textbf{Cuadrice Dublu Riglate în Arhitectură}}
\author{Andreiana Bogdan Mihail, Ghiță Vlăduț Adrian, Scânteie Alexandru Ioan}
\date{\today}

\begin{document}

\maketitle
\thispagestyle{empty}
\newpage

\tableofcontents
\newpage

\section{Introducere}

Cuadricele dublu riglate reprezintă o fascinantă întâlnire între matematică și arhitectură, demonstrând cum principiile geometrice avansate pot fi traduse în structuri reale de o frumusețe și funcționalitate remarcabile. Aceste suprafețe matematice, caracterizate prin proprietatea unică că prin fiecare punct al lor trec două drepte distincte conținute integral pe suprafață, au devenit fundamentul unor capodopere arhitecturale din întreaga lume.

De la Sagrada Família a lui Antoni Gaudí la modernista Gară din Predeal, de la Turnul Portului Kobe la arena Saddledome din Calgary, aceste forme geometrice demonstrează că matematica nu este doar un exercițiu teoretic, ci este capabilă să genereze spații care inspiră și funcționează în perfectă armonie cu necesitățile umane.

\section{Fundamentele Matematice}

\subsection{Definiția Cuadricelor Dublu Riglate}

O suprafață este considerată \textbf{dublu riglată} dacă prin fiecare punct al său trec două drepte distincte care sunt în întregime conținute în acea suprafață. Aceste drepte formează două familii distincte de generatoare rectilinii, cu proprietatea fundamentală că:

\begin{itemize}
    \item Oricare două generatoare din aceeași familie nu se intersectează și sunt necoplanare
    \item Oricare două generatoare din familii diferite sunt coplanare și se intersectează într-un singur punct
\end{itemize}

În geometria diferențială, există doar trei tipuri de suprafețe dublu riglate: planul, paraboloidul hiperbolic și hiperboloidul cu o pânză. Aceste două din urmă sunt principalele cuadrice dublu riglate utilizate în arhitectură.

\subsection{Hiperboloidul cu o Pânză}

\subsubsection{Definiția și Ecuația Canonică}

Hiperboloidul cu o pânză este locul geometric al punctelor din spațiu ale căror coordonate $(x, y, z)$ satisfac ecuația canonică:

\begin{equation}
\frac{x^2}{a^2} + \frac{y^2}{b^2} - \frac{z^2}{c^2} = 1
\end{equation}

unde $a$, $b$, $c$ sunt numere reale strict pozitive, reprezentând parametrii care determină forma hiperboloidului.

\subsubsection{Proprietăți Geometrice}

Hiperboloidul cu o pânză prezintă caracteristici geometrice remarcabile:

\begin{itemize}
    \item Este o suprafață conexă și necompactă
    \item Prezintă o îngustare caracteristică la mijloc, cunoscută sub numele de "gât"
    \item Se extinde la infinit în ambele direcții ale axei $z$
    \item Două generatoare rectilinii concurente din familii diferite determină planul tangent la hiperboloid în punctul lor de intersecție
\end{itemize}

\subsubsection{Generatoarele Rectilinii}
Ecuațiile celor două familii de generatoare rectilinii ale hiperboloidului pot fi exprimate parametric, demonstrând că întreaga suprafață poate fi generată prin mișcarea unei drepte în spațiu conform unor reguli geometrice precise. De exemplu, o familie de generatoare poate fi dată de sistemul:
\[
\begin{cases}
    \lambda \left( \frac{x}{a} - \frac{z}{c} \right) = 1 - \frac{y}{b} \\
    \frac{x}{a} + \frac{z}{c} = \lambda \left( 1 + \frac{y}{b} \right)
\end{cases}
\]
unde $\lambda$ este un parametru real.
\subsection{Paraboloidul Hiperbolic}

\subsubsection{Definiția și Ecuația Canonică}

Paraboloidul hiperbolic, cunoscut popular ca "șa de cal" datorită formei sale caracteristice, este definit prin ecuația:

\begin{equation}
\frac{x^2}{a^2} - \frac{y^2}{b^2} = z
\end{equation}

Această suprafață se caracterizează prin curburi de semne opuse în direcții perpendiculare, creând forma distinctivă de șa.

\subsubsection{Proprietăți Geometrice}

Paraboloidul hiperbolic prezintă următoarele caracteristici importante:

\begin{itemize}
    \item Punctul de origine $(0,0,0)$ este un punct de șa (punct critic de tip hiperbolic).
    \item Intersecțiile cu plane paralele cu planul $xOz$ sunt parabole (dacă $b \neq 0$).
    \item Intersecțiile cu plane paralele cu planul $yOz$ sunt parabole (dacă $a \neq 0$).
    \item Intersecțiile cu plane paralele cu planul $xy$ ($z=k$, $k \neq 0$) sunt hiperbole. Dacă $k=0$, intersecția este formată din două drepte.
    \item Toate generatoarele dintr-o aceeași familie sunt paralele cu un același plan director.
\end{itemize}

\subsubsection{Avantaje Constructive}

Proprietatea fundamentală a paraboloidului hiperbolic de a fi compus din segmente de dreaptă îl face extrem de atractiv pentru aplicații arhitecturale:

\begin{itemize}
    \item Construcție facilitată prin utilizarea exclusivă a grinzilor drepte
    \item Economie de material și simplificarea proceselor de fabricație
    \item Posibilitatea prefabricării elementelor structurale
    \item Distribuție optimă a încărcărilor pe suprafața curbată
\end{itemize}

\section{Aplicații în Arhitectura Modernă}

\subsection{Avantajele Arhitecturale ale Cuadricelor Dublu Riglate}

Utilizarea cuadricelor dublu riglate în arhitectură oferă beneficii multiple:

\begin{itemize}
    \item \textbf{Eficiență structurală}: Distribuția optimă a forțelor și rezistența superioară
    \item \textbf{Economia constructivă}: Folosirea elementelor prefabricate standard
    \item \textbf{Expresie estetică}: Forme organice și dinamice care se integrează armonios în peisaj
    \item \textbf{Funcționalitate}: Crearea de spații vaste fără suport intermediar
\end{itemize}

\subsection{Studii de Caz Semnificative}

\subsubsection{Gara din Predeal, România (1966-1968)}

Gara din Predeal reprezintă un exemplu remarcabil de experimentare cu suprafețele dublu riglate în arhitectura românească din anii 1960. Proiectul a fost realizat de \textbf{arhitecții Irina Rosetti și Ilie Dumitrescu}, cu inginerii \textbf{Mircea Mihăilescu și Ildikó Bucur Horváth}. Studiile pentru forma acoperișului au început în toamna anului 1966, iar clădirea a fost inaugurată în 1968.

\textbf{Caracteristici tehnice:}
\begin{itemize}
    \item Locație: Predeal, la altitudinea de 1.032 m (cel mai înalt punct din rețeaua CFR)
    \item Structură: Acoperișul în formă de paraboloid hiperbolic asimetric, realizat ca învelitoare subțire de beton
    \item Plan: Romboidal, acoperind clădirea gării cu o formă organică
    \item Context internațional: Face parte dintr-o tendință globală de experimentare cu pânze subțiri de beton, promovată prin IASS (International Association for Shell and Spatial Structures)
\end{itemize}

Această realizare se înscrie într-un context internațional de experimentare arhitecturală, alături de proiecte similare precum gara Ochota din Varșovia sau pavilionul izvorului 24 din stațiunea Olănești.

\subsubsection{Sagrada Família, Barcelona, Spania (1882-prezent)}

Antoni Gaudí a utilizat extensiv structuri hiperboloidale în această capodoperă arhitecturală, mai evident după 1914. Hiperboloidele se regăsesc în:

\begin{itemize}
    \item Coloanele principale care se ramifică ca copacii către tavan
    \item Bolțile navei centrale, transeptului și absidei
    \item Designul ferestrelor care creează efecte dramatice de lumină și umbră
    \item Turnurile care se înalță spre cer în forme spiralate
\end{itemize}

\textbf{Inovații tehnice ale lui Gaudí:}
\begin{itemize}
    \item Dezvoltarea unui sistem de calculare a forțelor folosind modele de sfoară și greutăți
    \item Eliminarea necesității contraforturilor clasice
    \item Crearea de spații luminoase cu acustică optimă
    \item Integrarea simbolismului religios în geometrie
\end{itemize}

\subsubsection{Turnul Portului Kobe, Japonia (1963)}

Cu o înălțime de 108 metri, acest turn de observație demonstrează aplicarea hiperboloidului în structuri înalte. Proiectat de compania Nikken Sekkei, designul este inspirat de tamburul tradițional japonez \textit{tsuzumi}. Turnul este pictat într-o culoare portocaliu ars și este înconjurat de 32 de stâlpi de oțel roșii.

\textbf{Avantaje structurale:}
\begin{itemize}
    \item Primul turn construit folosind o structură de zăbrele țevi
    \item Rezistență superioară la vânt comparativ cu structurile cilindrice
    \item Stabilitate seismică optimă pentru condițiile din Japonia
    \item Economia de material prin utilizarea structurii riglate
    \item Vizibilitate excelentă din toate direcțiile
\end{itemize}

\subsubsection{Scotiabank Saddledome, Calgary, Canada (1983)}

Arena sportivă cu acoperișul iconic în formă de paraboloid hiperbolic, cunoscută pentru designul său distinctiv inspirat de șeaua calului. Proiectată de Graham McCourt Architects, arena a câștigat multiple premii arhitecturale.

\textbf{Realizări tehnice:}
\begin{itemize}
    \item Acoperire a unei suprafețe de peste 20.000 m² fără coloane interioare
    \item Structură hiperbolică paraboloidă inversată, construită din panouri prefabricate de beton ușor
    \item Adaptare optimă la condițiile climatice extreme ale Calgary
    \item Eficiență termică superioară pentru o arenă de hockey
    \item Capacitate de 19.289 spectatori cu vizibilitate perfectă
\end{itemize}

\subsection{Arhitecți și Ingineri Vizionari}

\subsubsection{Antoni Gaudí (1852-1926)}
Arhitectul catalan este recunoscut pentru explorarea profundă a geometriei și a formelor naturale în arhitectură. Gaudí a înțeles intuitiv și a aplicat principiile suprafețelor riglate pentru a crea structuri stabile, eficiente și de o frumusețe organică. Lucrările sale, în special cele din ultima perioadă, demonstrează o măiestrie în utilizarea hiperboloizilor și paraboloizilor hiperbolici.

\subsubsection{Félix Candela (1910-1997)}
Arhitect și inginer spaniol-mexican, Candela a fost un pionier în proiectarea și construcția structurilor subțiri de beton armat ("cascarones"), multe dintre ele bazate pe forma paraboloidului hiperbolic. Lucrările sale, precum Restaurantul Los Manantiales din Xochimilco, sunt exemple emblematice ale eleganței și eficienței structurale a acestor forme.

\subsubsection{Experimente Arhitecturale în România și Contextul Internațional}
În anii '60 și '70, arhitecții și inginerii români au participat la tendințele internaționale de explorare a noilor forme structurale, în special a pânzelor subțiri de beton. Proiecte precum Gara din Predeal se înscriu în acest context, demonstrând o aliniere la curentele moderniste care căutau expresivitate și eficiență prin geometrii complexe. Aceste eforturi au fost adesea susținute de instituții de învățământ și cercetare tehnică din țară, contribuind la dezvoltarea unui limbaj arhitectural modern specific.

\section{Impactul asupra Arhitecturii Contemporane}

\subsection{Tehnologii Moderne de Proiectare și Construcție}
Dezvoltarea tehnologiilor moderne a facilitat și a extins aplicarea cuadricelor dublu riglate și a altor forme geometrice complexe:
\begin{itemize}
    \item \textbf{Proiectarea Parametrică și Modelarea 3D (BIM)}: Permit arhitecților și inginerilor să exploreze, să analizeze și să optimizeze geometrii complexe cu o precizie și o flexibilitate fără precedent.
    \item \textbf{Fabricația Digitală (CNC, Tăiere cu Laser, Imprimare 3D)}: Facilitează producția exactă a componentelor structurale și a cofrajelor complexe.
    \item \textbf{Materiale Avansate}: Betonul de înaltă performanță, materialele compozite și aliajele ușoare permit realizarea unor forme și mai îndrăznețe și eficiente.
    \item \textbf{Analiza Structurală Avansată (FEA)}: Permite simularea detaliată a comportamentului structural, asigurând siguranța și optimizarea designului.
\end{itemize}

\subsection{Sustenabilitatea și Eficiența Energetică}

Cuadricele dublu riglate contribuie la arhitectura sustenabilă prin:

\begin{itemize}
    \item Reducerea consumului de material prin optimizarea structurală
    \item Îmbunătățirea eficienței termice prin forme aerodinamice
    \item Maximizarea luminii naturale prin eliminarea suporturilor intermediare
    \item Durabilitatea superioară a structurilor optimizate geometric
\end{itemize}

\section{Concluzii}

Cuadricele dublu riglate demonstrează puterea matematicii de a genera soluții elegante pentru provocările arhitecturale complexe. De la ecuațiile abstracte ale hiperboloidului și paraboloidului hiperbolic la realizări concrete ca Sagrada Família sau Gara din Predeal, aceste forme geometrice continuă să inspire arhitectura contemporană.

Succesul acestor aplicații rezidă în înțelegerea profundă a principiilor matematice fundamentale și în capacitatea de a le traduce în soluții constructive practice. Arhitecții vizionari ca Antoni Gaudí și echipele de proiectare din întreaga lume au demonstrat că matematica nu constrânge creativitatea, ci o eliberează, oferind instrumente puternice pentru crearea de spații care îmbină funcționalitatea cu frumusețea.

În era digitală actuală, cu posibilitățile oferite de modelarea computerizată și fabricația de precizie, cuadricele dublu riglate devin din ce în ce mai accesibile arhitecților contemporani, promițând noi capodopere care vor continua să demonstreze frumusețea eternă a matematicii aplicate în arhitectură.

\section*{Bibliografie}

\begin{enumerate}[label=\arabic*.]
    \item Bucur-Horváth, Ildikó (2012). \textit{Gara din Predeal. Observații și date suplimentare}. Prezentare IASS Symposium 2012, Seoul, Korea.
    \item Dumitrescu, Ilie și Rosetti, Irina (1968). \textit{Proiectul Gării Predeal}. Institutul de Proiectări Cluj, București.
    \item Fuchs, Dmitry B. și Tabachnikov, Serge (2007). \textit{Mathematical Omnibus: Thirty Lectures on Classic Mathematics}. American Mathematical Society.
    \item Graham McCourt Architects (1983). \textit{Scotiabank Saddledome: Documentație de proiect}. Calgary, Canada.
    \item Hilbert, David și Cohn-Vossen, Stephan (1999). \textit{Geometry and the Imagination}. Chelsea Publishing, New York.
    \item International Association for Shell and Spatial Structures (1959-prezent). \textit{Proceedings of IASS Symposiums}. Diverse volume și locații.
    \item Mihăilescu, Mircea (1966-1968). \textit{Studii structurale pentru Gara Predeal}. Catedra Construcții de beton armat, Institutul Politehnic Cluj.
    \item Nikken Sekkei Company (1963). \textit{Kobe Port Tower: Technical Documentation}. Kobe, Japan.
    \item Weisstein, Eric W. (1999). \textit{Doubly Ruled Surface}. MathWorld - A Wolfram Web Resource.
    \item Zeppelin, e- (2019). \textit{Istoria acum: Gara de călători Predeal (1967-1968)}. Disponibil la: https://e-zeppelin.ro/istoria-acum-gara-de-calatori-predeal-1967-1968/
\end{enumerate}

\end{document} 